\documentclass[12pt, a4paper]{book}
\begin{document}
The standard model of particle physics
\begin{equation}
    \mathcal{L}_{SM} = -\frac{1}{4}F_{\mu\nu}F^{\mu\nu} + i\overline{\psi}\slashed{D}\psi + \psi_iy_{ij}\psi_j\phi + h.c. + \abs{D_\mu\phi}^2 - V(\phi)
\end{equation}
The elegant equation that can explain phenomena in nature with great precision. It consists of three \textit{symmetry groups}. The group explaining electromagnetism $U(1)$, the group describing the weak force $SU(2)_L$ and the group describing the strong force $SU(3)_C$. 
In this chapter we will delve into each of these gruops, the spontaneous symmetry breaking phenomena that gives mass to particles, as well as how to calculate the cross section of processes.\\
\\The theory of this section is mainly based of Peskin's and Schroeder's "An Introduction to Quantum Field Theory" \cite{Peskin:1995ev} and Thomson's "Modern Particle Physics" \cite{THOMSON}.

\clearpage
\section{Quantum Electrodynamics}
In the begining there was nothing; \textit{then God said, “Let there be light,” and there was light.} This lead us to the first part of the Standard Model, Quantum Electrodynamics, or QED for short,
\begin{equation}\label{eq:QED}
    \mathcal{L}_{QED} = \bar{\psi}\left(i\slashed{D} -m\right)\psi -\frac{1}{4}F^{\mu\nu}F_{\mu\nu}
\end{equation}
using the notation $\slashed{D}\equiv \gamma^\mu D_\mu$ where $iD_\mu = i\partial_\mu -eA_\mu$ is the covariant derivative and $F_{\mu\nu}=\partial_\mu A_\nu - \partial_\nu A_\mu$ is the electromagnetic field tensor\footnote{Taking $\partial_\mu F^{\mu\nu} = 0$ gives all of Maxwells equations.}. $A_\mu$ is the vector potential of the Lagrangian.\\ 
\\The above Lagrangian is part of a $U(1)$ symmetry group, meaning that there is only one unique gauge freedom, this is the rotation of phase angle of the field $\theta$. The $m$ is the mass of the \textit{Dirac} particle described by the Dirac field $\psi$.

\section{Quantum Chromodynamics}
While QED is expressed by the Lagrangian in Eq. \ref{eq:QED}, which is part of a $U(1)$ symmetry group, it can still be described by a more general Lagrangian using \textit{Yang Mills} theory
\begin{equation}\label{eq:YM_lag}
    \mathcal{L}_{YM} = \bar{\psi}\left(i\slashed{D} -m\right)\psi -\frac{1}{4}(F^a_{\mu\nu})^2
\end{equation}
which looks almost identical to the QED Lagrangian, the differences being that the Dirac Lagrangian\footnote{Left part of the QED Lagrangian $ \bar{\psi}\left(i\slashed{D} -m\right)\psi $} part of the equation changes the covariant derivative
$$
iD_\mu = i\partial_\mu -eA_\mu\rightarrow iD_\mu = i\partial_\mu -gA^a_\mu t^a
$$
where $e$ is replaced for a more general scale factor $g$, and this new $t^a$ factor represents which local gauge symmetry we have on the field,
$$
\psi \rightarrow e^{i \alpha^a t^a + \mathcal{O}(\alpha^2)} \psi
$$ 
The other difference is that the Dirac equation becomes
\begin{equation}\label{eq:YM_eq}
    \partial^\mu F^a_{\mu\nu} +g\epsilon^{ijk}A^{j\mu}F^k_{\mu\nu} = -g\bar{\psi}\gamma_\nu\frac{\sigma^a}{2}\psi \rightarrow -g\bar{\psi}\gamma_\nu t^a\psi
\end{equation}
The important thing to note about the Yang Mills is that the covariant derivative can be \textit{non Abelian}, meaning that 
$$
\left[D_\mu,D_\nu\right] = -igF_{\mu\nu}^at^a
$$
where it is non Abelian if the right had side of the equation is non zero. 
\section{Electroweak theory and the Brout-Englert-Higgs Mechanism}
\section{Adding it all up}
When combibing all these groups we get spontaneous symmmetry breaking resulting in the Brout-Englert-Higgs Mechanism. The whole lagrangian is of the form
$$
U(1)_Y\otimes SU(2)_L\otimes SU(3)_C \Rightarrow 
$$
\begin{equation}
    \mathcal{L}_{SM} = -\frac{1}{4}F_{\mu\nu}F^{\mu\nu} + i\overline{\psi}\slashed{D}\psi + \psi_iy_{ij}\psi_j\phi + h.c. + \abs{D_\mu\phi}^2 - V(\phi)
\end{equation}
where 
$$
V(\phi)=-\mu^2\phi^*\phi + \frac{\lambda}{2}(\phi^*\phi)^2
$$
is the Higgs potential.\\
\\All of this is great at explaining what we know so far
\section{S-marix expansion}

\end{document}